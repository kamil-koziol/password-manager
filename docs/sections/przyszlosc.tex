\section{Przyszłość menedżerów haseł}

\subsection{WebAuthn}

Web Authentication (WebAuthn) to otwarty standard W3C, który umożliwia uwierzytelnianie użytkowników bez używania tradycyjnych haseł. Wykorzystywana jest kryptografia asymetryczna, która polega na zastosowaniu kluczy publicznych i prywatnych. Aby skorzystać z WetAuthn, użytkownik musi najpierw zarejestrować swoje  klucze w usłudze internetowej, którą chce używać. Bezpieczne klucze mogą być zapisane na fizycznych urządzeniach, takich jak klucze USB lub karty inteligentne, lub mogą być generowane na podstawie biometrii. Klucz publiczny jest zaszyfrowany i przechowywany na serwerze, a klucz prywatny znajduje się lokalnie na urządzeniu użytkownika. Co istotne, standard ten jest obsługiwany przez większość dostępnych przeglądarek internetowych, co umożliwia jego szerokie zastosowanie.

Podczas procesu uwierzytelniania, usługa internetowa wysyła zapytanie do urządzenia użytkownika, aby potwierdzić tożsamość użytkownika. Użytkownik potwierdza swoją tożsamość, używając bezpiecznego klucza, który jest przechowywany na urządzeniu. Urządzenie użytkownika następnie przekazuje odpowiednie informacje zwrotne do usługi internetowej, potwierdzając tożsamość użytkownika.
\section{Menedżery}
Istnieje wiele innych menedżerów haseł, które oferują różne funkcje i poziomy bezpieczeństwa. Wybór konkretnego menedżera haseł zależy od indywidualnych preferencji użytkownika i wymagań dotyczących bezpieczeństwa danych.

\subsection{Wybrane menadzery}
Lista wybranych menadzerow, które zostaną przedstawione:
\begin{itemize}
    \item LastPass,
    \item 1Password,
    \item NordPass,
    \item KeePassXC,
    \item BitWarden.
\end{itemize}

\subsubsection{LastPass}

Jest to jedna z najpopularniejszych aplikacji do przechowywania haseł. LastPass umożliwia przechowywanie haseł, kart kredytowych i innych danych w chmurze, a także generowanie mocnych haseł i automatyczne wypełnianie formularzy.
\subsubsection{Wady}

\subsubsection{Zalety}

\subsection{1Password}
Wieloplatformowy menedżer haseł, którego producentem jest przedsiębiorstwo AgileBits, możliwia przechowywanie haseł, kart kredytowych i innych danych w chmurze. 1Password jest płatynym oprogramowaniem zamkniętym na licencji shareware. Aplikacja korzysta z funkcji haszującej PBKDF2 i szyfruje dane przy użyciu 256-bitowego algorytmu AES. W celu zwiększenia bezpieczeństwa stosowana jest również weryfikacja dwuetapowa.
\subsubsection{Wady}
\begin{itemize}
    \item Płatny
\end{itemize}
\subsubsection{Zalety}
\begin{itemize}
    \item Synchronizacja z chumrą
    \item Możliwość przechowywania dokumentów, obrazów i innych plików
    \item W przypadku zgubienia Tajnego klucza nie można wygenerować nowego i konieczne jest utworzenie nowego konta
\end{itemize}

\subsection{NordPass}
\subsubsection{Wady}
\subsubsection{Zalety}

\subsection{KeePassXC}
Został stworzony przez KeePassXC Team, jest bezplatnym narzedziem open source, ktore dziala na licencji GLP 2.0 i 3.0. Powstal on jako widelec KeePassX, ktory przestal byc rozwijany.
Baza danych jest szyfrowana za pomocą algorytmu AES256 lub szyfru blokowego Twofish, a nastepnie jest uzyta funkcja haszowana. Mozna dodatkowo zabezpieczyc baze plikiem-kluczem wypelnionym dowolna iloscia losowych bajtow lub uzywajac YubiKey. 
\subsubsection{Wady}
\begin{itemize}
    \item Brak wbudowanej synchronizacji z chmura,
    \item Zlozony interfejs,
    \item Brak dedykowanej aplikacji mobilnej.
\end{itemize}

\subsubsection{Zalety}
\begin{itemize}
    \item Open source,
    \item Dwuskladnikowe uwierzytelnianie,
    \item Funkcja Auto-Type, automatycznie wpisuje nazwę użytkownika i hasło w formularzach.
\end{itemize}


\subsection{BitWarden}

Bezpłatny menedżer haseł, który oferuje szyfrowanie end-to-end i dwuskładnikową autoryzację. Bitwarden umożliwia także przechowywanie danych karty kredytowej i wypełnianie formularzy.

\subsubsection{Wady}
\subsubsection{Zalety}
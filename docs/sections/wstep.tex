\section{Wstęp}
Raport ma na celu przedstawienie konceptu menadzerow hasel, przeglad dostepnych rozwiazan oraz prezentacje wlasnej implementacji. Dokument zostal stworzony na potrzeby przedmiotu Wprowadzenie do Cyberbezpeczenstwa. 

\subsection{Czym jest password menager?}
Menedżer haseł to program komputerowy lub narzędzie, które pozwala użytkownikom zarządzać hasłami i bezpiecznie je zapisywać. Oferuje szybkie i bezpieczne rozwiązanie do przechowywania haseł do różnych kont internetowych, w tym poczty e-mail, sieci społecznościowych, bankowości i innych witryn lub usług.
\subsubsection{Przechowywanie haseł}
Menedżer haseł bezpiecznie przechowuje wszystkie hasła w zaszyfrowanej bazie danych. Zamiast zapamiętywać wiele haseł, wystarczy zapamiętać tylko jedno hasło główne, aby uzyskać dostęp do menedżera haseł.
\subsubsection{Generowanie haseł}
Menedżery haseł często zawierają funkcję generowania haseł, która może tworzyć silne, unikalne hasła dla każdego konta. Pomaga to zwiększyć bezpieczeństwo poprzez stosowanie złożonych i losowych haseł, które są trudne do odgadnięcia lub złamania przez hakerów.
\subsubsection{Autouzupełnianie i wypełnianie formularzy}
Menedżery haseł mogą automatycznie wypełniać dane logowania na stronach internetowych lub w aplikacjach, eliminując potrzebę ręcznego wprowadzania nazw użytkowników i haseł. Funkcja ta oszczędza czas i zmniejsza ryzyko błędów podczas logowania.
\subsubsection{Dostęp międzyplatformowy}
Menedżery haseł są zazwyczaj dostępne na wielu platformach, w tym na komputerach stacjonarnych, laptopach, smartfonach i tabletach. Umożliwia to użytkownikom dostęp do haseł z różnych urządzeń i synchronizację danych między platformami.
\subsubsection{Bezpieczna synchronizacja danych}
Menedżery haseł często oferują opcje synchronizacji w chmurze, umożliwiając bezpieczną synchronizację bazy danych haseł na wielu urządzeniach. Gwarantuje to, że hasła są aktualne i dostępne wszędzie tam, gdzie są potrzebne.

\subsection{Założenia}
Menedżery haseł wykorzystują zaawansowane techniki szyfrowania, aby zapobiec nieautoryzowanemu dostępowi do danych haseł.
Użytkownicy powinni wybrać hasło główne, które jest unikalne, złożone i trudne do odgadnięcia.
Bardzo ważne jest, aby oprogramowanie do zarządzania hasłami było aktualne, aby korzystać z najnowszych funkcji bezpieczeństwa i poprawek.
Jeśli menedżer haseł to oferuje, użytkownicy powinni włączyć dodatkowe środki bezpieczeństwa, takie jak uwierzytelnianie dwuskładnikowe (2FA).
W zależności od dostępnych funkcji, menedżery haseł mogą również przechowywać inne poufne informacje, takie jak dane karty kredytowej, chronione notatki lub dane osobowe.

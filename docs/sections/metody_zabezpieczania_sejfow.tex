\section{Metody zabezpieczania sejfów}

\subsection{Metody szyfrowania sejfu}

Menedżery haseł zapewniają bezpieczne przechowywanie haseł i innych poufnych danych dzięki zastosowaniu zaawansowanych algorytmów szyfrowania z których najczęściej stosowane to:

\begin{itemize}
    \item Advanced Encryption Standard (AES) – jest to jeden z najbezpieczniejszych i najpopularniejszych algorytmów szyfrowania stosowanych przez menedżery haseł. AES działa na blokach danych o długości 128 bitów i wykorzystuje klucz szyfrowania o długości 128, 192 lub 256 bitów.
    \item Blowfish – jest to szybki i skuteczny algorytm szyfrowania stosowany przez menedżery haseł. Blowfish działa na blokach danych o długości 64 bitów i wykorzystuje klucz szyfrowania o długości od 32 do 448 bitów.
    \item Twofish – to inny popularny algorytm szyfrowania stosowany przez menedżery haseł. Twofish działa na blokach danych o długości 128 bitów i wykorzystuje klucz szyfrowania o długości 128, 192 lub 256 bitów.
  \end{itemize}

\subsection{AES}
\subsubsection{Historia}
AES (Advanced Encryption Standard) jest to symetryczny szyfr blokowy oparty na algorytmach Rijndaela, storzonych w 1998 w ramach konkursu gdzie zostały zaprezentowane Instytucji NIST. Na ich podstawie wybranych algorytów powstał AES będący następcą algorytmu DES.
\subsubsection{Działanie}
AES działa na stałej długości bloków (128 bitów) i używa tajnego klucza, który może przyjąć długości 128, 192, 256 bitów. Cechuje się wysoką wydajnością zarówno w przypadku sprzętu komputerowego, jak i oprogramowania. AES opiera się na zasadzie znanej jako sieć substytucji-permutacji. W odróżnieniu od swojego poprzednika, czyli algorytmu DES, AES nie korzysta z Sieci Feistela. Algorytm operuje na macierzy bajtów o wymiarach 4×4, nazywaną stanem. Jednak niektóre wersje algorytmu dysponują większym rozmiarem bloku oraz dodatkowymi kolumnami w macierzy.

Rozmiar klucza używany w algorytmie określa liczbę powtórzeń transformacji, które przekształcają dane wejściowe (czyli tekst jawny) w dane wyjściowe (szyfrogram). Liczba cykli powtórzeń jest następująca:

10 cykli powtórzeń dla klucza 128-bitowego.
12 cykli powtórzeń dla klucza 192-bitowego.
14 cykli powtórzeń dla klucza 256-bitowego.
Wszystkie rundy składają się z kilku kroków, z których każdy rozłożony jest na cztery podobne etapy.

Wykonując operacje w odwrotnej kolejności używając tego samego klucza, można przekształcić szyfrogram z powrotem w tekst jawny.
\subsection{Blowfish}
\subsubsection{Historia}
Blowfish to symetryczny algorytm szyfrowania blokowego opracowany przez Bruce'a Schneiera w 1993 roku. Bruce Schneier stworzył Blowfish jako szyfrowanie prywatne, dostępne publicznie i niepodlegające opatentowaniu. Algorytm został opublikowany w celu zapewnienia efektywnego i bezpiecznego szyfrowania.

\subsubsection{Działanie}
Blowfish działa na blokach danych o długości 64 bitów. Algorytm Blowfish rozpoczyna się od inicjalizacji klucza o długości od 32 do 448 bitów. Klucz jest podzielony na podklucze, które są wykorzystywane w kolejnych rundach szyfrowania. Algorytm generuje zestaw podkluczy inicjalizacyjnych na podstawie głównego klucza. Podklucze są przechowywane w wewnętrznej tablicy i używane do przekształcania danych wejściowych. 

Blowfish składa się z 16 rund szyfrowania. W każdej rundzie dane wejściowe są poddawane serii operacji, które obejmują operacje XOR, zamianę bitów, dodawanie i mieszanie kluczy. Rundy szyfrowania są powtarzane na blokach danych, aż osiągnie się pożądaną ilość iteracji. Każda runda używa podklucza z tablicy podkluczy. 

Po zakończeniu rund szyfrowania, dane wejściowe są zaszyfrowane i zwracane jako tekst szyfrowany. Proces odszyfrowywania w Blowfish jest odwróceniem procesu szyfrowania. Otrzymany tekst szyfrowany jest poddawany przeciwnym operacjom, aż otrzymano oryginalne dane wejściowe.

\subsection{Twofish}
\subsubsection{Historia}
Twofish to algorytm szyfru blokowego z kluczem symetrycznym, który został zaprojektowany przez Bruce'a Schneiera, Johna Kelseya, Douga Whitinga, Davida Wagnera, Chrisa Halla i Nielsa Fergusona w 1998 roku. Jest to następca algorytmu szyfrowania Blowfish.
\subsubsection{Działanie}
Twofish działa na blokach danych o stałej długości (128 bitów) i używa tajnego klucza do szyfrowania i deszyfrowania danych. Długość klucza może wynosić od 128 do 256 bitów, co czyni go bardziej bezpiecznym niż Blowfish.

Algorytm Twofish wykorzystuje strukturę sieci Feistela, która jest powszechnie stosowaną strukturą w szyfrach blokowych. Polega on na podzieleniu danych wejściowych na dwa równej wielkości bloki, a następnie zastosowaniu serii rund do bloków przy użyciu harmonogramu klucza.

Podczas każdej rundy, nieliniowa funkcja substytucji (S-box) jest stosowana do każdego bloku, aby wprowadzić zamieszanie, a liniowa transformacja (macierz MDS) jest stosowana do każdego bloku, aby wprowadzić dyfuzję. Harmonogram klucza jest używany do generowania zestawu kluczy okrągłych z oryginalnego tajnego klucza, które są używane w każdej rundzie do modyfikacji danych wejściowych.

Jedną z kluczowych mocnych stron Twofish jest jego zależne od klucza S-boxy, które sprawiają, że jest bardziej odporny na niektóre ataki, takie jak kryptoanaliza różnicowa. Ma również silny efekt lawinowy, co oznacza, że niewielka zmiana w danych wejściowych lub kluczu powoduje znaczącą zmianę w danych wyjściowych.

Został on przyjęty przez kilka organizacji zajmujących się standardami bezpieczeństwa, w tym Narodowy Instytut Standardów i Technologii (NIST) oraz Międzynarodową Organizację Normalizacyjną (ISO). Szyfr Twofish nie zostal opatentowany. 


Wszystkie powyższe algorytmy szyfrowania są uważane za bardzo bezpieczne i wykorzystywane przez wiele menedżerów haseł. Ponadto, menedżery haseł często stosują dodatkowe zabezpieczenia, takie jak hasła główne, dwuskładnikową autoryzację (np. TOTP, UbiKey) i szyfrowanie końcowe do ochrony danych użytkownika.

\subsection{Metody hashowania haseł głównych}

\begin{itemize}
    \item PBKDF2 (Password-Based Key Derivation Function 2) – jest to algorytm hashowania, który wykorzystuje iteracyjny proces szyfrowania w celu zwiększenia bezpieczeństwa hasła głównego. PBKDF2 wykorzystuje funkcję skrótu, która generuje wartość hasha na podstawie hasła użytkownika, a następnie przeprowadza iteracyjny proces szyfrowania w celu uzyskania klucza szyfrującego. Proces ten jest powtarzany wielokrotnie, co zwiększa trudność w łamaniu hasła głównego.
    \item Bcrypt – jest to inny popularny algorytm hashowania, który jest bardzo bezpieczny i trudny do złamania. Bcrypt wykorzystuje funkcję skrótu, która generuje wartość hasha na podstawie hasła użytkownika i soli, która jest dodawana do hasha w celu zwiększenia bezpieczeństwa. Bcrypt jest również iteracyjny i wykorzystuje różne parametry, takie jak koszt, który określa, jak długo ma trwać proces hashowania.
    \item Scrypt – jest to algorytm hashowania, który został opracowany specjalnie do ochrony haseł przed atakami typu brute force i słownikowymi. Scrypt wykorzystuje funkcję skrótu, która generuje wartość hasha na podstawie hasła użytkownika i soli, a następnie przeprowadza iteracyjny proces szyfrowania w celu uzyskania klucza szyfrującego. Proces ten jest powtarzany wielokrotnie, co utrudnia atakującym łamanie hasła głównego.
\end{itemize}
